\chapter{Úvod}

\setcounter{page}{1}

\begin{chapterabstract}
	Tato kapitola slouží jako úvod do tématiky hlasování v poslanecké sněmovně. Vysvětlíme si, k čemu slouží a jak funguje hlasování v poslanecké sněmovně. Poté popíši, jakým způsobem je průběh a výsledek hlasování poskytnut veřejnosti. Nakonci uvedu motivaci k vytvoření mobilní aplikace pro sledování průběhu hlasování.
\end{chapterabstract}

\section{Poslanecká sněmovna}
Základní prvky politického systému ČR představuje prezident, vláda, Parlament a ústavní soud. Ústava ČR dělí moc na zákonodárnou – Parlament, který je složen z Poslanecké sněmovny a Senátu, výkonnou – prezident, vláda a státní zastupitelství a soudní – Ústavní soud a obecné soudy. \cite{Husek2019-p40}

Parlament České republiky se skládá ze dvou komor – Poslanecké sněmovny (dolní komora) a Senátu (horní komora). Poslanecká sněmovna se skládá z 200 poslanců a je volena na čtyři roky na základě poměrného volebního systému.\cite{Husek2019-p40}

\section{Hlasování v poslanecké sněmovně}
Komora PS je usnášeníschopné, pokud je přítomna alespoň jedna třetina jejích členů. K přijetí usnesení (tzn. ke schválení zákona) je nutný souhlas nadpoloviční většiny přítomných poslanců, pokud ústava nestanoví jinak. \cite{Husek2019-p40}

Proces návrhu a schvalování zákona je komplexní a řídí se podle určitých pravidel. Pro účely této práce se však budu zabývat pouze schvalovacím procesem v PS. Více informací ohledně procesu přijímání zákonů lze najít na https://www.psp.cz/sqw/hp.sqw?k=173.
	

\section{Webový portál psp.cz}
Hlavním zdrojem pro výsledky a průběhy hlasování je oficiální webový portál psp.cz. Tento portál poskytuje mnoho informací, pro účely této práce však budu čerpat především strojově zpracovatelná data, která budou nutná pro implementaci mobilní aplikace.

\section{Motivace pro tuto práci}
Webový portál obsahuje všechny informace o hlasováních, může však pro některé uživatele být nepřehledný. Motivací pro tuto práci je poskytnout uživatelovi mobilní aplikaci pro sledování výsledků hlasování, s intuitivním uživatelským rozhraním a snadným způsobem, jak se dostat k detailnější, informacím jako např. stenoprotokol, jak hlasovaly kluby a poslanci, výsledky hlasování v minulých volebních obdobích.