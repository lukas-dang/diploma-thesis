\chapter{Úvod}

\setcounter{page}{1}

\begin{chapterabstract}
	Tato kapitola slouží jako úvod do tématiky hlasování v poslanecké sněmovně. Na začátku popíši, k čemu slouží a jak funguje hlasování v poslanecké sněmovně. Poté popíši, jakým způsobem je průběh a výsledek hlasování poskytnut veřejnosti. Nakonci uvedu motivaci k vytvoření mobilní aplikace pro sledování průběhu hlasování.
\end{chapterabstract}


\section{Hlasování v poslanecké sněmovně}
Poslanecká sněmovna, jakožto jedna z komor Parlamentu České Republiky, je jedna ze státních orgánů, které schvalují zákony. Tyto zákony mohou ovlivňovat každodenní život obyvatel ČR, je tedy vhodné sledovat, které zákony jsou poslaneckou sněmovnou schvalovány a jak hlasovaly jednotlivé poslanecké kluby či poslanci. Tyto poznatky nám ´mohou pomoct při rozhodování o tom, pro který poslanecký klub a pro které poslance hlasovat do poslanecké sněmovny v příštích volbách.

\section{Webový portál psp.cz}
Hlavním zdrojem pro informace o výsledcích a průběhu všech hlasování je oficiální webový portál psp.cz. Tento portál poskytuje mnoho informací, pro účely této práce však budou stačit strojově zpracovatelná data, která budou potřebná pro implementaci mobilní aplikace.

\section{Motivace pro tuto práci}
Webový portál obsahuje všechny informace o hlasováních, může však být pro některé uživatele nepřehledný. Aby se uživatel dostal k nejnovějšímu hlasování, musí přijíjt na webový portál, najít, kde se seznam hlasování najít a proklikat se k němu. Pokud si chce uživatel navíc přečíst další informace jako např. stenoprotokol, jak hlasovaly kluby a poslanci, nebo podívat se na výsledky hlasování v minulých volebních obdobích, musí tyto sekce hledat a proklikat se k nim. Motivací pro tuto práci je ulehčit práci uživatelovi, a poskytnout mu intuitivní rozhraní, pomocí kterého se ke všem nejdůležitějším informacím o hlasováních dostane na pár kliknutí a nemusí informace dlouho hledat.