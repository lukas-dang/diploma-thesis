\chapter{Návrh}

\setcounter{page}{1}

\section{Uživatelské rozhraní}

\subsection*{Seznam hlasování}

Na obrázku (\ref{fig:vote_list}) je návrh obrazovky pro seznam hlasování. Ta je složena z hlavičky, seznamu hlasování a dolní navigace. Hlavička obsahuje titul identifikující danou obrazovku,  aktuálně nastavené volební období a tlačítko pro vyhledávání hlasování. Titul slouží pro snazší orientaci v aplikaci. Volební období slouží pro snazší kontrolu, jaké volební období je aktuálně nastaveno. Jednotlivá hlasování v seznamu jsou rozdělena do jednotlivých boxů. Každý box obsahuje krátký popis návrhu zákona, datum a čas hlasování, výsledek hlasování znázorněný ikonkou a textem, a indikátor pro kliknutí na dané hlasování. Dolní navigace pak slouží pro navigaci mezi hlavními obrazovkami, tj. mezi obrazovkou pro seznam hlasování, seznam poslanců a nastavení.

Kliknutím na tlačítko pro vyhledávání se zobrazí vyhledávací pole (\ref{fig:vote_list_search}), které slouží pro vyhledávání seznamu hlasování podle jejich popisu. Do pole uživatel zadává klíčová slova. Pole obsahuje placeholder text, tlačítko pro smazání textu a tlačítko pro schování vyhledávacího pole.%

\subsection*{Detail hlasování}

Obrazovka pro detail hlasování (\ref{fig:vote_details_party_votes}) je složena z hlavičky a obsahu. Hlavička obsahuje tlačítko pro navigaci zpět. Obsah je rozdělen do dvou tabů, mezi kterými lze navigovat pomocí dvou tlačítek pod hlavičkou. V prvním tabu se nachází obecné údaje o daném návrhu zákona a výsledích jeho hlasování. Tyto údaje zahrnují popis návrhu zákona, datum a čas hlasování, odkaz na oficiální portál se stenoprotokolem a tabulku s informacemi ohledně toho, jak se hlasovalo. Každý typ hlasování (ano, ne, nepřihlášen, omluven, zdržel se) je popsáno textově i pomocí ikonky.

Druhý tab (\ref{fig:vote_details_party_votes}) obsahuje seznam poslaneckých klubů v daném volebním období, rozdělených do boxů. V každém boxu je název klubu, jeho logo, pokud je k dispozici, a indikátor pro expandování boxu pro zobrazení informací o tom, jak daný klub a jeho členové hlasovali (\ref{fig:vote_details_party_votes}). Expandovaný box obsahuje navíc tabulku se statistikou hlasování jako v prvním tabu, ale pro konkrétní poslanecký klub. Pod tabulkou je seznam členů klubu a to, jak pro daný návrh zákona hlasovali. Výsledek hlasování členů je znázorněno ikonkou.

\subsection*{Seznam poslanců}

Obrazovka pro seznam poslanců (\ref{fig:member_list_search}) vypadá podobně jako obrazovka pro seznam hlasování. Obsahuje hlavičku s titulem, aktuálně nastaveným volebním obdobím a tlačítkém pro vyhledávání, seznam poslanců a dolní navigaci. Poslanci jsou rozděleny do boxů, které obsahují profilovou fotku poslance, jméno a příjmení, volební kraj, poslanecký klub a indikátor pro kliknutí. Kliknutím se dostaneme na obrazovku s detailem daného poslance. 

Pomocí vyhledávacího pole (\ref{fig:member_list_search}) lze poslance filtrovat podle jejich jména a příjmení.

\subsection*{Detail poslance}

Na obrazovce s detailem poslance (\ref{fig:member_details_general}) můžeme vidět údaje o daném poslanci a informace o tom, jak hlasoval pro návrhy zákonů. Obrazovka je rozdělena na hlavičku a obsah. Hlavička obsahuje tlačítko pro navigaci zpět. Obsah je rozdělen do dvou tabů. První z tabů obsahuje údaje o daném poslanci, tj. profilovou fotku, jméno a příjmení, datum a narození, datum mandátu poslance, poslanecký klub, do kterého přísluší, a volební kraj. Druhý tab (\ref{fig:member_details_votes}) obsahuje údaje o tom, jak daný poslanec hlasoval pro jednotlivé návrhy zákonů. Návrh je podobný jako návrh seznamu hlasování, obsahuje navíc údaj o tom, jak hlasoval daný poslanec.

\subsection*{Nastavení}
	
Na obrazovce pro nastavení (\ref{fig:settings_opened}) je seznam nastavení dané aplikace. V době psaní této práce obsahuje pouze jedno možné nastavení, kterým je volební období. Nastavení obsahuje ikonku znázorňující typ nastavení (v tomto případě volební období), název nastavení a text nastaveného volebního obdobi. Kliknutím na toto nastavení naskočí okno (\ref{fig:settings_opened}) se seznamem volebních období. Po zvolení volebního období uživatel může kliknout na tlačítko Uložit, kterým se dané volební období nastaví, nebo Zrušit, čímž se zruší aktuální výběr v seznamu.

\section{REST API}

Mobilní aplikace komunikuje s backendem pomocí REST API. Tato kapitola popisuje endpointy této API, její vstupy a výstupy.

\subsection*{HTTP hlavička}

\begin{itemize}
	\item prev - Odkaz na předchozí stránku. Null, pokud aktuální stránka je první.
	\item next - Odkaz na následující stránku. Null, pokud aktuální stránka je poslední.
	\item last - Odkaz na poslední stránku.
	\item self - Odkaz na aktuální stránku.
\end{itemize}

\subsection*{Endpointy}

\begin{lstlisting}
GET /api/app
\end{lstlisting}

\noindent Vrací následující informace o stavu aplikace (\ref{fig:app}):

\begin{itemize}
	\item Volební roky
\end{itemize}

\vspace{10px}

\begin{lstlisting}
GET /api/vote
\end{lstlisting}

\noindent Vrací seznam hlasování s následujícími informace (\ref{fig:vote}):
\begin{itemize}
	\item identifikátor hlasování
	\item datum a čas hlasování
	\item popis hlasování
	\item výsledek hlasování (A - přijato, R - zamítnuto, jinak zmatečné hlasování)
\end{itemize}

\vspace{10px}

\begin{lstlisting}
GET /api/vote{id}
\end{lstlisting}

\noindent Vrací následující informace o detailu hlasování (\ref{fig:vote-1}):
\begin{itemize}
	\item identifikátor hlasování
	\item datum a čas hlasování
	\item popis hlasování
	\item výsledek hlasování (A - přijato, R - zamítnuto, jinak zmatečné hlasování)
	\item URL odkaz na příslušný stenoprotokol
	\item počet hlasování pro
	\item počet hlasování proti
	\item počet nepřihlášených
	\item počet omluvených
	\item počet zdržených
	\item volební rok
\end{itemize}

\vspace{10px}

\begin{lstlisting}
GET /api/party/vote/{id}
\end{lstlisting}

\noindent Vrací následující informace o hlasováních poslaneckých klubů v daném hlasování (\ref{fig:party-vote-1}):

\begin{itemize}
	\item název klubu
	\item URL odkaz na logo klubu
	\item identifikátor hlasování
	\item výsledky hlasování klubu
	\item výsledky hlasování členů klubu
\end{itemize}

\vspace{10px}

\begin{lstlisting}
GET /api/member
\end{lstlisting}

\noindent Vrací seznam poslanců s následujícími informacemi:

\begin{itemize}
	\item identifikátor
	\item jméno a příjmení
	\item poslanecký klub
	\item URL odkaz na profilovou fotku
	\item volební kraj
	\item volební rok
\end{itemize}

\vspace{10px}

\begin{lstlisting}
GET /api/member/{id}
\end{lstlisting}

\noindent Vrací následující informace o detailu poslance:

\begin{itemize}
	\item identifikátor
	\item jméno a příjmení
	\item pohlaví (M - muž, Ž - ostatní)
	\item poslanecký klub
	\item začátek mandátu
	\item konec mandátu
	\item datum narození
	\item volební kraj
	\item URL odkaz na profilovou fotku, pokud existuje, jinak null
	\item volební rok
\end{itemize}

\vspace{10px}

\begin{lstlisting}
GET /api/member/1/vote
\end{lstlisting}

\noindent Vrací následující informace o hlasováních daného poslance:

\begin{itemize}
\item obecné informace o hlasování
\item výsledek hlasování poslance (A - ano, B nebo N - ne, C - zdržel se, F - nehlasoval, @ - nepřihlášen, K - zdržel se / nehlasoval)
\end{itemize}

\section{Databázové datové struktury}

Struktura agency.

\begin{table}[!h]\centering
	\caption[Struktura agency]{Struktura agency}\label{table:agency}
	\begin{tabular}{|l|l|p{6cm}|}\hline
		Název	& Typ	& Popis	\tabularnewline \hline \hline
		\texttt{id}		& \texttt{int}	& identifikátor orgánu		\tabularnewline \hline
		\texttt{abbreviation}		& \texttt{varchar(255)}	& zkratka názvu orgánu		\tabularnewline \hline
		\texttt{end\textunderscore date}		& \texttt{date(255)}	& datum zániku orgánu		\tabularnewline \hline
		\texttt{name}		& \texttt{varchar(255)}	& název orgánu		\tabularnewline \hline
		\texttt{start\textunderscore date}		& \texttt{date}	& datum založení orgánu		\tabularnewline \hline
		\texttt{type\textunderscore id}		& \texttt{int}	& identifikátor typu orgánu 		\tabularnewline \hline
	\end{tabular}
\end{table}

Struktura agency\textunderscore type.

\begin{table}[!h]\centering
	\caption[Struktura agency\textunderscore type]{Struktura agency\textunderscore type}\label{table:agency_type}
	\begin{tabular}{|l|l|p{6cm}|}\hline
		Název	& Typ	& Popis	\tabularnewline \hline \hline
		\texttt{id}		& \texttt{int}	& identifikátor typu orgánu		\tabularnewline \hline
		\texttt{name}		& \texttt{varchar(512)}	& název typu orgánu \tabularnewline \hline
		\texttt{superior\textunderscore agency\textunderscore type\textunderscore id}		& \texttt{int}	& identifikátor nadřazeného typu orgánu \tabularnewline \hline
	\end{tabular}
\end{table}

Struktura excuse.

\begin{table}[!h]\centering
	\caption[Struktura excuse]{Struktura excuse}\label{table:excuse}
	\begin{tabular}{|l|l|p{6cm}|}\hline
		Název	& Typ	& Popis	\tabularnewline \hline \hline
		\texttt{member\textunderscore id}		& \texttt{int}	& identifikátor poslance, který je omluven		\tabularnewline \hline
		\texttt{date} & \texttt{date}	& datum, kdy je poslanec omluven \tabularnewline \hline
		\texttt{start\textunderscore time}		& \texttt{time}	& čas, od kterého byl poslanec omluven \tabularnewline \hline
		\texttt{end\textunderscore time}		& \texttt{time}	& čas, do kterého byl poslanec omluven \tabularnewline \hline
		\texttt{election\textunderscore year}		& \texttt{int}	& první rok volebního období \tabularnewline \hline
	\end{tabular}
\end{table}

Struktura member.

\begin{table}[!h]\centering
	\caption[Struktura member]{Struktura member}\label{table:member}
	\begin{tabular}{|l|l|p{6cm}|}\hline
		Název	& Typ	& Popis	\tabularnewline \hline \hline
		\texttt{id}		& \texttt{int}	& identifikátor poslance, který je omluven		\tabularnewline \hline
		\texttt{date\textunderscore of\textunderscore birth} & \texttt{date}	& datum narození \tabularnewline \hline
		\texttt{election\textunderscore region}		& \texttt{varchar(255)}	& volební kraj \tabularnewline \hline
		\texttt{election\textunderscore year}		& \texttt{int}	& první rok volebního období \tabularnewline \hline
		\texttt{gender}		& \texttt{varchar(255)}	& pohlaví \tabularnewline \hline
		\texttt{member\textunderscore from} & \texttt{date} & datum začátku členství \tabularnewline \hline
		\texttt{member\textunderscore to} & \texttt{date} & datum konce členství \tabularnewline \hline
		\texttt{name} & \texttt{varchar(255)} & jméno \tabularnewline \hline
		\texttt{person\textunderscore id} & \texttt{int} & identifikátor osoby \tabularnewline \hline
		\texttt{photo\textunderscore url} & \texttt{varchar(255)} & URL profilové fotky \tabularnewline \hline
		\texttt{party\textunderscore election\textunderscore year} & \texttt{int} & první rok volebního období \tabularnewline \hline
		\texttt{party\textunderscore party\textunderscore id} & \texttt{int} & identifikíátor poslaneckého klubu, jehož je členem \tabularnewline \hline
	\end{tabular}
\end{table}

Struktura member\textunderscore vote.

\begin{table}[!h]\centering
	\caption[Struktura member\textunderscore vote]{Struktura member\textunderscore vote}\label{table:membe_vote}
	\begin{tabular}{|l|l|p{6cm}|}\hline
		Název	& Typ	& Popis	\tabularnewline \hline \hline
		\texttt{result}		& \texttt{varchar(255)}	& jak hlasoval poslanec\tabularnewline \hline
		\texttt{member\textunderscore id}		& \texttt{int}	& jak identifikátor poslance\tabularnewline \hline
		\texttt{vote\textunderscore id}		& \texttt{int}	& identifikátor hlasování\tabularnewline \hline
	\end{tabular}
\end{table}

Struktura membership.

\begin{table}[!h]\centering
	\caption[Struktura membership]{Struktura membership}\label{table:membership}
	\begin{tabular}{|l|l|p{6cm}|}\hline
		Název	& Typ	& Popis	\tabularnewline \hline \hline
		\texttt{end\textunderscore date}		& \texttt{datetime}	& datum a čas konce zařazení\tabularnewline \hline
		\texttt{agency\textunderscore id}		& \texttt{int}	& identifikátor orgánu\tabularnewline \hline
		\texttt{person\textunderscore id}		& \texttt{int}	& identifikátor osoby \tabularnewline \hline
		\texttt{start\textunderscore date}		& \texttt{datetime}	& datum a čas začátku zařazení \tabularnewline \hline
	\end{tabular}
\end{table}

Struktura party.

\begin{table}[!h]\centering
	\caption[Struktura party]{Struktura party}\label{table:party}
	\begin{tabular}{|l|l|p{6cm}|}\hline
		Název	& Typ	& Popis	\tabularnewline \hline \hline
		\texttt{abbreviation} & \texttt{varchar(255)} & zkratka pro název klubu\tabularnewline \hline
		\texttt{name} & \texttt{varchar(255)}	& název klubu\tabularnewline \hline
		\texttt{election\textunderscore year}		& \texttt{int}	& první rok volebního období \tabularnewline \hline
		\texttt{party\textunderscore id}		& \texttt{int}	& identifikátor klubu \tabularnewline \hline
	\end{tabular}
\end{table}

Struktura vote.

\begin{table}[!h]\centering
	\caption[Struktura vote]{Struktura vote}\label{table:vote}
	\begin{tabular}{|l|l|p{6cm}|}\hline
		Název	& Typ	& Popis	\tabularnewline \hline \hline
		\texttt{date\textunderscore time} & \texttt{datetime} & datum a čas hlasování\tabularnewline \hline
		\texttt{description} & \texttt{varchar(255)}	& popis hlasování\tabularnewline \hline
		\texttt{election\textunderscore year}		& \texttt{int}	& první rok volebního období \tabularnewline \hline
		\texttt{excused\textunderscore count}		& \texttt{int}	& počet omluvených \tabularnewline \hline
		\texttt{logged\textunderscore off\textunderscore count}		& \texttt{int}	& počet nepřihlášených \tabularnewline \hline
		\texttt{meeting\textunderscore number}		& \texttt{int}	& bod hlasování \tabularnewline \hline
		\texttt{no\textunderscore count}		& \texttt{int}	& počet hlasování proti \tabularnewline \hline
		\texttt{refrained\textunderscore count}		& \texttt{int}	& počet zdržených \tabularnewline \hline
		\texttt{result}		& \texttt{varchar(255)}	& výsledek hlasování \tabularnewline \hline
		\texttt{steno\textunderscore protocol\textunderscore url}		& \texttt{varchar(255)}	& stenoprotokol \tabularnewline \hline
		\texttt{yes\textunderscore count}		& \texttt{int}	& počet hlasování pro \tabularnewline \hline
		\texttt{number}		& \texttt{int}	& číslo hlasování \tabularnewline \hline
		\texttt{id}		& \texttt{int}	& identifikátor hlasování \tabularnewline \hline
	\end{tabular}
\end{table}