\chapter{Funkční a nefunkční požadavky}

\setcounter{page}{1}

\begin{chapterabstract}
	V této kapitole popisuji funkční a nefunkční požadavky na mobilní aplikaci a backendu. Funkční požadavky specifikují funkcionality, které by měl daný software poskytovat. Nefunkční požadavky určují omezení kladená na daný software.
\end{chapterabstract}

\section{Funkční požadavky}
V této podkapitole uvádím funkční požadavky pro mobilní aplikaci (\ref{table:func_req_app}) a backend (\ref{table:func_req_be}). Ke každému požadavku uvádím identifikátor pro pozdější odkazování k požadavku.

\def\arraystretch{1.5}
\begin{longtable}{|l|p{9cm}|} \hline
	\multicolumn{2}{|l|}{\textbf{Funkční požadavky pro mobilní aplikaci}} \\ \hline
	
	\textbf{ID požadavku} & \textbf{Popis požadavku} \\ \hline
	
	FP\textunderscore01	& Aplikace bude umět zobrazit seznam výsledků hlasování. Kromě výsledku budou jednotlivá hlasování v seznamu obsahovat také název hlasování, a datum a čas, kdy bylo odhlasováno. \\ \hline
	
	FP\textunderscore02	& Aplikace bude umět zobrazit detail hlasování. Detail hlasování bude obsahovat název, datum a čas, odkaz na stenoprotokol a celkovou statistiku hlasování. Celkovou statistikou hlasování rozumíme počet hlasování pro ano, ne, nepřihlášeno, omluveno a zdrženo od hlasování. Dále bude obsahovat to, jak v daném hlasování hlasovaly jednotlivé poslanecké kluby a členy těchto klubů. \\ \hline
	
	FP\textunderscore03	& Aplikace bude umět zobrazit seznam členů poslanecké sněmovny. Prvky v tomto seznam budou obsahovat stručné informace o daném poslanci. Tyto informace budou obsahovat jméno a příjmení, volební kraj, název klubu a profilovou fotku. \\ \hline
	
	FP\textunderscore04	& Aplikace bude umět zobrazit detail poslance. Detail poslance bude obsahovat jméno a příjemní, datum narození, profilovou fotku, datum nabytí statusu poslance, poslanecký klub a volební kraj. Dále bude obsahovat seznam výsledků hlasování a to, jak v nich hlasoval daný poslanec\\ \hline
	
	FP\textunderscore05	& Aplikace bude poskytovat možnost nastavení volební období, při kterém se nastaví hlasování a poslanci daného volebního období.\\ \hline
	
	FP\textunderscore06	& Aplikace bude poskytovat možnost vyhledávání hlasování podle jeho názvu.\\ \hline
	
	FP\textunderscore07	& Aplikace bude poskytovat možnost vyhledávání poslance / poslankyně podle jeho / jejího jména.\\ \hline
	
	\caption{Funkční požadavky pro mobilní aplikaci.}
	\label{table:func_req_app}
\end{longtable}

\def\arraystretch{1.5}
\begin{longtable}{|l|p{9cm}|} \hline
	\multicolumn{2}{|l|}{\textbf{Funkční požadavky pro back-end}} \\ \hline
	\textbf{ID požadavku} & \textbf{Popis požadavku} \\ \hline
	
	FP\textunderscore01	& Backend bude prostřednictvím volně dostupného rozhraní poskytovat mobilní aplikaci všechna data, která bude potřebovat.  \\ \hline
	
	\caption{Funkční požadavky pro back-end.}
	\label{table:func_req_be}
\end{longtable}

\section{Nefunkční požadavky}

V této podkapitole uvádím nefunkční požadavky pro mobilní aplikaci (\ref{table:nonfunc_req_app}) a backend (\ref{table:nonfunc_req_be}).


\def\arraystretch{1.5}
\begin{longtable}{|l|p{9cm}|} \hline
	\multicolumn{2}{|l|}{\textbf{Nefunkční požadavky pro mobilní aplikaci}} \\ \hline
	\textbf{ID požadavku} & \textbf{Popis požadavku} \\ \hline
	
	NP\textunderscore 00	& Aplikace nebude provádět výpočetně náročná zpracování dat z důvodu šetření aplikace. Toto zpracování bude delegováno na backend. \\ \hline
	
	NP\textunderscore01	& Aplikace bude data stahovat pouze po spuštění aplikace z důvodu úspory dat a šetření BE. \\ \hline

	NP\textunderscore 02 & 
	Aplikace bude podporovat pouze čas v ČR, jelikož cílí na hlasování o zákonech v ČR. Tj. pokud používáme aplikaci v zahraničí, zobrazené časy budou stále lokální vzhledem k ČR. \\ \hline

	NP\textunderscore 03 & Aplikace bude mít jednoduché a intuitivní uživatelské rozhraní. \\ \hline
	
	NP\textunderscore 04 & Aplikace bude fungovat na zařízeních s OS Android 5.1 a výš pro cílení většího množství uživatelů. \\ \hline
	
	NP\textunderscore 05 & Aplikace nebude sbírat uživatelská data. \\ \hline
	
	NP\textunderscore 06 & Aplikace bude používat architekturu doporučenou v oficiální dokumenteci Androidu (https://developer.android.com/topic/architecture). \\ \hline
	
	NP\textunderscore07	& Backend bude zdrojová data zpracovávat tak, aby se nemusely zpracovávat v mobilní aplikaci. Zpracování dat na straně klienta by mohlo aplikaci zpomalit. Část dat, jejichž zpracování bude trvat příliš dlouho (např. půl dne), lze zpracovávat za běhu, pokud to nebude trvat příliš dlouho, a tím zhoršovat user experience uživatele. Rozhodnutí o to, kdy budou data zpracvávána při startu backendu a kdy až za běhu, je na pocitu programátora user testování mobilní aplikace. \\ \hline
	
	\caption{Nefunkční požadavky pro mobilní aplikaci.}
	\label{table:nonfunc_req_app}
\end{longtable}

\def\arraystretch{1.5}
\begin{longtable}{|l|p{9cm}|} \hline
	\multicolumn{2}{|l|}{\textbf{Nefunkční požadavky pro back-end}} \\ \hline
	\textbf{ID požadavku} & \textbf{Popis požadavku} \\ \hline
	
	NP\textunderscore 01 & Backend bude data stahovat z oficiální portálu PSP. \\ \hline
	
	NP\textunderscore 02 & Backend bude data stažená z portálu PSP transformovat do databázového modelu, který bude popsán v kapitole o návrhu aplikace a backendu. Cílem je, aby data byla předzpracovaná a připravená pro rychlý přistup z mobilní aplikace. \\ \hline
	
	NP\textunderscore 03 & Backend bude ztransformovaná data ukládat do databáze pro jejich perzistenci a připravu pro použití mobilní aplikací kdykoliv. \\ \hline
	
	NP\textunderscore 04 & Backend bude ztransformovaná a uložena data vystavovat prostřednictvím REST API. \\ \hline
	
	NP\textunderscore 03 & Backend bude každý den stahovat nová data z portálu PSP a aktualizovat databázi. \\ \hline
	
	NP\textunderscore 05 & Backend bude data vystavovat ve formatu JSON. \\ \hline	
	
	\caption{Nefunkční požadavky pro back-end.}
	\label{table:nonfunc_req_be}
\end{longtable}
