\chapter{Analýza zdrojových dat}
\label{ch:analysis_data}

\begin{chapterabstract}
	V této kapitole budou analyzována zdrojová data PSP, která budou použita pro backend.
\end{chapterabstract}

\section{Zdrojové soubory}
\label{sec:source-data}

Zdrojové soubory se nachází v souborech ve formátu zip, které lze stáhnout na webu PSP \cite{psp-data}. Pro tuto práci budou použity následující zdrojové soubory:

\begin{itemize}
	\item \textbf{poslanci.zip} - Obsahuje datové soubory pro poslance a osoby, jejich zařazení do orgánů \linebreak a orgány jako takové.
	\item \textbf{hl-XXXXps.zip} -- Obsahuje datové soubory pro návrhy zákonů a výsledky jejich \linebreak hlasování, a výsledky hlasování poslanců.
\end{itemize}

\section{Formát dat}
Data v datových souborech jsou poskytována ve formátu UNL \cite{psp-data}, tj.:

\begin{itemize}
	\item Každý řádek v souboru odpovídá jednom řádku v databázi.
	\item Oddělovačem je znak roury (|).
	\item Pokud je sloupec prázdný, je jeho hodnota typu \lstinline|null|.
	\item V sloupcích jsou používány tzv. escape sekvence k zápisu speciálních znaků s úvodním znakem \ (backslash) následovaný znakem.
\end{itemize}

\noindent Další informace o datech:

\begin{itemize}
	\item Kódování dat je \lstinline|windows-1250|.
	\item Pokud bude strunktura dat doplňována, budou nové sloupce přidávány na konec.
\end{itemize}

\noindent Kódování \lstinline|windows-1250| obsahuje mimo jiné všechny znaky z české abecedy. Je tedy třeba dbát na správně nastavené kódování při parsování nebo ukládání do databáze, aby datům vráceným mobilní aplikaci nechyběly např. háčky a čárky. Zároveň při parsování UNL souborů je potřeba brát ohled na přidávání nových sloupců nakonec.

\section{Aktualizace}
Web PSP \cite{psp-data} uvádí, že data obsahují úplný stav a že rozdílové aktualizace nejsou poskytovány. To je pro použití neideální, jelikož při aktualizaci databáze je pak potřeba z webu stáhnout znovu všechna zdrojová data a poté buď sestavit znovu celou databázi nebo zjistit rozdílový stav a na základě něho aktualizovat stávající databázi.

\section{Datové typy}
Na webu PSP \cite{psp-data} jsou poskytovány informace o datových typech sloupců. V tabulce \ref{table:data_types} jsou uvedeny ty datové typy, které jsou použity v rámci této práce.

\begin{longtable}{|l|p{9cm}|} \hline
	\multicolumn{2}{|l|}{\textbf{Typy dat sloupců v tabulkách}} \\ \hline
	\textbf{Typ} & \textbf{Popis} \\ \hline
	
	int	& integer \\ \hline
	
	char(X)		& textový řetězec, s blíže neuvedenou délkou
	\\ \hline

	date	& datum, ve formátu DD.MM.YYYY
	\\ \hline	
	
	datetime(year to hour)		& datum a čas, do úrovně hodin, ve formátu YYYY-MM-DD HH
	
	\\ \hline
	
	datetime(hour to minute)		& čas, ve formátu HH:MM
	\\ \hline
	
	\caption{Datové typy sloupců v datových souborech \cite{psp-data}}
	\label{table:data_types}
\end{longtable}

\section{Licence}

Data jsou poskytována bezplatně, využití dat je podmíněno uvedením zdroje dat a případně datem zpracování dat \cite{psp-data}. V mobilní aplikaci bude uveden zdroj.

\section{Datové soubory}

Zde budou vypsány použité datové soubory ve zdrojových souborech uvedených v podkapitole \ref{sec:source-data} a jejich stručný popis. Soubory budou popsány detailněji v následující sekci \ref{sec:db-tables}. Ze zdrojového souboru \lstinline|poslanci.zip| budou použity následující datové soubory:

\begin{itemize}
	\item \textbf{organy} -- Reprezentuje orgány ve státní správě. Zahrnuje parlamenty všech volebních období a poslanecké kluby. Data budou použita pro nalezení poslaneckých klub.
	
	\item \textbf{poslanec} -- Reprezentuje poslance, včetně těch z předchozích volebních období. Data budou použita pro získání všech poslanců v určitém volebním období.
	
	\item \textbf{osoby} -- Reprezentuje osobu. Obsahuje osobní údaje jako jméno, příjmení a datum narození. Data budou použita pro zjištění osobních údajů o poslanci.
	
	\item \textbf{zarazeni} -- Obsahuje zařazení osoby do nějakého orgánu. Data budou použita pro získání členů poslaneckého klubu.
\end{itemize}

\noindent Ze zdrojových souborů \lstinline|hl-XXXX.zip|, kde XXXX je volební rok, budou používány následující datové soubory:

\begin{itemize}
	\item \textbf{hl\_hlasovani} -- Reprezentuje výsledky hlasování o návrhů zákonů. Data budou použita pro získání informací o návrhu zákona a výsledcích jeho hlasování.
	
	\item \textbf{hl\_poslanec} -- Reprezentuje výsledek hlasování poslance o návrhu zákona. Data budou použita pro získání výsledků hlasování poslance o návrhu zákona.
	
	\item \textbf{omluvy} -- Reprezentuje časově ohraničené omluvy z jednání a hlasování. Data budou použita pro získání počtu omluvených. 
\end{itemize}

\section{Tabulky}
\label{sec:db-tables}

Na \ref{fig:class-diagram} lze vidět diagram zdrojových dat, s kterými bude dále v práci pracováno. Obsahuje typy dat, jejich atributy, datové typy atributů a vazby mezi typy. Atributy identifikující daný typ jsou zvýrazněny tučným písmem. Následovat bude výčet použitých tabulek s popisem atributů. Pro stručnost budou u tabulek uvedeny pouze atributy, které jsou v rámci práce používány. Poznatky v textu jsou převážně z \textbf{analýzy obsahu datových souborů} a nejsou na stránkách PS nijak zdokumentovány.

\subsection*{organy}
Tabulka \textbf{organy} \ref{table:organy} bude použita pro nalezení poslaneckých klubů v určitém volebním období. Všechny poslanecké kluby mají hodnotu atributu \textbf{id\_typ\_organu} rovnou 1. Poslanecké kluby v daném volebním období lze nalézt kontrolou, zda hodnoty atributů \textbf{od\_organ} a \textbf{do\_organ} spadají mezi hodnotami stejných atributů PSP. PSP je totiž také orgánem a lze ji najít prostřednictvím hodnoty atributu \textbf{id\_organ}. První PSP má identifikátor 165. PSP v každém následujícím volebním období má hodnotu identifikátoru o jednu větší než PSP v předchozím volebním období. Tedy např. PSP ve volebním období 2017 až 2021 má identifikátor 172. Dále pokud hodnota atributu \textbf{do\_organ} je rovna \lstinline|null|, pak jde o aktuální orgán.

\begin{center}
	\begin{longtable}{|l|l|p{9cm}|}
		\caption{Tabulka organy} \label{table:organy} \\
		
		\hline 
		
		\multicolumn{3}{|l|}{\textbf{Tabulka organy}} \\
		
		\hline 
		
		\multicolumn{1}{|l|}{\textbf{Sloupec}} & \multicolumn{1}{l|}{\textbf{Typ}} & \multicolumn{1}{l|}{\textbf{Použití a vazby}} \\ 
		
		\endhead
		
		\hline 
		
		id\textunderscore organ & int & Identifikátor orgánu \\
		
		\hline 
		
		id\textunderscore typ\textunderscore organu & int & Identifikátor typu orgánu \\
		
		\hline 
		
		zkratka & char(X) & Zkratka orgánu \\
		
		\hline 
		
		nazev\textunderscore organu\textunderscore cz	 & char(X)	 & Název orgánu v češtině
		\\
		
		\hline 
		
		od\textunderscore organ & date & Datum ustavení orgánu
		\\
		
		\hline 
		
		do\textunderscore organ & date & Datum ukončení orgánu
		\\
		
		\hline 
		
	\end{longtable}
\end{center}

\subsection*{osoby}

Tabulka \textbf{osoby} \ref{table:osoby} bude použita pro získání osobních údajů o poslancích. Tabulka však nezahrnuje pouze poslance, ale i další osoby senátu. Pro zjištění, kterého poslance reprezentuje daná osoba, je potřeba použít tabulku \textbf{zarazeni}. Zde je zajímavé datum narození, které je 1. 1. 1900, pokud je neznámo. Zobrazovat toto datum v mobilní aplikaci by nebylo ideální. V kapitole \linebreak o návrhu REST API bude popsána transformace tohoto data na jinou hodnotu.

\begin{center}
	\begin{longtable}{|l|l|p{9cm}|}
		\caption{Tabulka osoby
		} \label{table:osoby} \\
		
		\hline 
		
		\multicolumn{3}{|l|}{\textbf{Tabulka osoby
		}} \\
		
		\hline 
		
		\multicolumn{1}{|l|}{\textbf{Sloupec}} & \multicolumn{1}{l|}{\textbf{Typ}} & \multicolumn{1}{l|}{\textbf{Použití a vazby}} \\ 
		
		\endhead
		
		\hline 
		
		id\textunderscore osoba & int & Identifikátor osoby \\
		
		\hline 
		
		jmeno & char(X) & Jméno \\
		
		\hline 
		
		prijmeni & char(X) & Příjmení \\
		
		\hline 
		
		narozeni & date & Datum narození, pokud neznámo, pak 1.1.1900. \\
		
		\hline 
		
		pohlavi & char(X) & Pohlaví, "M" jako muž, ostatní hodnoty žena \\
		
		\hline 
		
		
	\end{longtable}
\end{center}

\subsection*{zarazeni}

Tabulka zařazení bude použita ke zjištění příslušnosti poslance do poslaneckého klubu. Zde je potřeba dát pozor na to, že atribut \textbf{id\_of} může reprezentovat buď tabulku \textbf{organ} (tu používáme) nebo tabulku \textbf{funkce} (tu nepoužíváme, a tudíž tu není uvedena) podle toho, zda je hodnota atributu \textbf{cl\_funkce} rovna 0 nebo 1. Tabulka \textbf{funkce} reprezentuje konkrétní funkci v daném orgánu. Pro účely této práce tato informace není potřeba, a proto tu tabulka \textbf{funkce} není uvedena. Je však potřeba zjistit příslušnost poslance do poslaneckého klubu, kterou lze zjistit kombinací tabulky \textbf{zarazeni} a \textbf{organy}. Důsledkem popsaného je to, že nás budou zajímat pouze zařazení \linebreak s hodnotou atributu \textbf{cl\_funkce} rovnou 0. Tím pádem identifikátor \textbf{id\_of} bude vždy referovat k tabulce \textbf{organy}. Tabulka \textbf{organy} obsahuje poslanecké kluby, což je přesně to, co potřebujeme. Dále pokud hodnota atributu \textbf{do\_o} je rovna \lstinline|null|, pak jde o aktuální zařazení.

\newpage

\begin{center}
	\begin{longtable}{|l|l|p{8cm}|}
		\caption{Tabulka zarazeni} 
		\label{table:zarazeni} \\
		
		\hline 
		
		\multicolumn{3}{|l|}{\textbf{Tabulka zarazeni}} \\
		
		\hline 
		
		\multicolumn{1}{|l|}{\textbf{Sloupec}} & \multicolumn{1}{l|}{\textbf{Typ}} & \multicolumn{1}{l|}{\textbf{Použití a vazby}} \\ 
		
		\endhead
		
		\hline 
		
		id\textunderscore osoba & int & Identifikátor osoby, viz osoba:id\textunderscore osoba \\
		
		\hline 
		
		id\textunderscore of & int & Identifikátor orgánu či funkce: pokud je zároveň nastaveno zarazeni:cl\textunderscore funkce == 0, pak id\textunderscore o odpovídá organy:id\textunderscore organ, pokud cl\textunderscore funkce == 1, pak odpovídá funkce:id\textunderscore funkce.
		\\
		
		\hline 
		
		cl\textunderscore funkce & int & Status členství nebo funkce: pokud je rovno 0, pak jde o členství, pokud 1, pak jde o funkci.
		\\
		
		\hline 
		
		od\textunderscore o & datetime(year to hour)	 & Zařazení od
		\\
		
		\hline 
		
		do\textunderscore o & datetime(year to hour)	 & Zařazení do
		\\
		
		\hline 
		
	\end{longtable}
\end{center}

\subsection*{poslanec}

Tabulka \textbf{poslanec} \ref{table:poslanec} slouží pro získání informací o všech poslancích. Pokud je někdo poslancem ve více volebních obdobích, pak bude v této tabulce mít více záznamů. Tedy dva poslanci mohou reprezentovat tutéž osobu, pokud jsou z různých volebních období. Dále pomocí atributu \textbf{id\_osoba} lze tabulku zjistit osobní údaje o poslanci. Dále pomocí atributu \textbf{id\_kraj} lze získat název volebního kraje. A pomocí atributu \textbf{id\_obdobi} lze zjisti volební období, do kterého přísluší. Hodnota atributu \textbf{id\_obdobi} totiž odpovídá identifikátoru některé z PSP. Např. pokud poslanec patří do PSP v prvním volebním období (orgán mající identifikátor 165), pak je hodnota atributu \textbf{id\_obdobi} rovna 165. Dále někteří poslanci mají profilové foto. Po analýze fotek poslanců na oficiálním portálu PSP byl zjištěn následující vzor pro URL fotky \linebreak poslance: \lstinline|https://www.psp.cz/eknih/cdrom/XXXXps/eknih/XXXXps/poslanci/iYYY.jpg|, kde \lstinline|XXXX| je identifikátor osoby (pozor, ne identikátor poslance) a \lstinline|YYY| je identifikátor PS (např. 165). Příklad: pokud bychom chtěli najít URL s fotkou poslance s identifikátorem 1659, zjistíme hodnoty atributů \textbf{id\_kraj} a \textbf{id\_obdobi}, a na základě nich vygenerujeme URL fotky.

\begin{center}
	\begin{longtable}{|l|l|p{9cm}|}
		\caption{Tabulka poslanec} 
		\label{table:poslanec} \\
		
		\hline 
		
		\multicolumn{3}{|l|}{\textbf{Tabulka poslanec}} \\
		
		\hline 
		
		\multicolumn{1}{|l|}{\textbf{Sloupec}} & \multicolumn{1}{l|}{\textbf{Typ}} & \multicolumn{1}{l|}{\textbf{Použití a vazby}} \\ 
		
		\endhead
		
		\hline 
		
		id\textunderscore poslanec & int & Identifikátor poslance \\
		
		\hline 
		
		id\textunderscore osoba & int & Identifikátor osoby, viz osoba:id\textunderscore osoba \\
		
		\hline 
		
		id\textunderscore kraj & int & Volební kraj, viz organy:id\textunderscore organu \\
		
		\hline 
		
		id\textunderscore obdobi & int & Volební období, viz organy:id\textunderscore organu \\
		
		\hline 
		
		foto & int & Pokud je rovno 1, pak existuje fotografie poslance. \\
		
		\hline 
		
	\end{longtable}
\end{center}

\subsection*{hl\textunderscore hlasovani}

Tabulka \textbf{hl\_hlasovani} obsahuje většinu potřebných informací o hlasováních o daném návrhu zákona. Čislo schůze a číslo hlasování budou použity pro sestavení URL pro stenoprotokol daného hlasování. Podle webu se od účinnosti novely jednacího řádu 90/1995 Sb. nerozlišuje zdržel se \linebreak a nehlasoval, tj. příslušné počty se sčítají. Z toho plyne, že by mobilní aplikaci měla u hlasováních uskutečněných před účinností této novely mít kolonku pro počet poslanců, kterí nehlasovali. \linebreak A u hlasováních uskutečněných po účinnosti této novely by tam daná kolonka již být neměla.

\begin{center}
	\begin{longtable}{|l|l|p{7cm}|}
		\caption{Tabulka hl\textunderscore hlasovani} 
		\label{table:hl_hlasovani} \\
		
		\hline 
		
		\multicolumn{3}{|l|}{\textbf{Tabulka hl\textunderscore hlasovani}} \\
		
		\hline 
		
		\multicolumn{1}{|l|}{\textbf{Sloupec}} & \multicolumn{1}{l|}{\textbf{Typ}} & \multicolumn{1}{l|}{\textbf{Použití a vazby}} \\ 
		
		\endhead
		
		\hline 
		
		id\textunderscore hlasovani & int & Identifikátor hlasování \\
		
		\hline 
		
		schuze & int & Číslo schůze
		\\
		
		\hline 
		
		cislo & int & Číslo hlasování
		\\
		
		\hline 
		
		datum & date & Datum hlasování
		\\
		
		\hline 
		
		čas & datetime(hour to minute)	 & Čas hlasování
		\\
		
		\hline 
		
		pro & int & Počet hlasujících pro
		\\
		
		\hline 
		
		proti & int & Počet hlasujících proti
		\\
		
		\hline 
		
		zdrzel & int & Počet hlasujících zdržel se, tj. stiskl tlačítko X
		\\
		
		\hline 
		
		nehlasoval & int & Počet přihlášených, kteří nestiskli žádné tlačítko
		\\
		
		\hline 
		
		prihlaseno & int & Počet přihlášených poslanců
		\\
		
		\hline 
		
		vysledek & char(X)	 & Výsledek: A -- přijato, R -- zamítnuto, jinak zmatečné hlasování
		\\
		
		\hline 
		
		nazev\textunderscore dlouhy & char(X)	 & Dlouhý název bodu hlasování
		\\
		
		\hline 
		
	\end{longtable}
\end{center}

\subsection*{hl\textunderscore poslanec}

Tabulka \textbf{hl\_poslanec} obsahue informace o tom, jak hlasoval který poslanec v rámci kterého hlasování. Výsledky 'B' a 'N' jsou interpretovány stejně, oba znamenají hlasování proti. Výsledek 'Fb se používalo před rokem 1995. Po roce 1995 má vždy hodnotu 0. Výsledek 'W' je velmi ojedinělý a v případě, že nastane, sečte se do počtu nepřihlášených. U výsledku 'K' se mi nepodařilo zjistit, co přesně znamená zdržel se/nehlasoval. Ve zdrojových datech se vysktuje spíš méně, ale častěji než výsledek 'W'. Na webu PSP je tento výsledek u jednoho hlasování v roce 2013 interpretován jako nehlasoval. Z toho lze usoudit pouze to, že 'K' může znamenat nehlasoval. Není však z toho jasné, kdy může znamenat zdržel se. Na výsledek hlasování má vliv pouze počet pro a proti, a tudíž bylo rozhodnuto, že tento výsledek bude mobilní aplikaci ignorován. 

\begin{center}
	\begin{longtable}{|l|l|p{9cm}|}
		\caption{Tabulka hl\textunderscore poslanec} 
		\label{table:hl_poslanec} \\
		
		\hline 
		
		\multicolumn{3}{|l|}{\textbf{Tabulka hl\textunderscore poslanec}} \\
		
		\hline 
		
		\multicolumn{1}{|l|}{\textbf{Sloupec}} & \multicolumn{1}{l|}{\textbf{Typ}} & \multicolumn{1}{l|}{\textbf{Použití a vazby}} \\ 
		
		\endhead
		
		\hline 
		
		id\textunderscore poslanec & int & Identifikátor poslance, viz poslanec:id\textunderscore poslanec
		\\
		
		\hline 
		
		id\textunderscore hlasovani & int & Identifikátor hlasování, viz hl\textunderscore hlasovani:id\textunderscore hlasovani
		\\
		
		\hline 
		
		vysledek & char(X) & Hlasování jednotlivého poslance. 'A' - ano, 'B' nebo 'N' -- ne, 'C' -- zdržel se, 'F' -- nehlasoval, '@' -- nepřihlášen, 'M' -- omluven, 'W' -- hlasování před složením slibu poslance, 'K' -- zdržel se/nehlasoval.
		\\
		
		\hline 
		
	\end{longtable}
\end{center}

\subsection*{omluvy}

Tabulka \textbf{omluvy} \ref{table:omluvy} zaznamenává časové ohraničení omluv poslanců z jednání Poslanecké sněmovny. Slouží pouze po spočtení počtu omluvených poslanců během hlasování. Podáváme-li se na statistiky o hlasování poskytnuté tabulkou \textbf{hl\_hlasovani}, uvidíme, že poskytuje počet hlasování pro, proti, a počet zdržených, ale již ne počet nepřihlášených a omluvených. Počet omluvených byl přidán až po 1995, kdy pouze nahradil počet poslanců, kteří nehlasovali. Počet nehlasujících poslanců tedy nepomůže k odvození počtu nepřihlášených a omluvených. Jediná další informace v tabulce \textbf{hl\_hlasovani} je tedy počet přihlášených. Intuitivně by člověk čekal, že doplněk pro počet přihlášených je počet nepřihlášených, ale není tomu tak. Když jsem od počtu poslanců (200) odečetl počet přihlášených, nikdy to nevycházelo s počty nepřihlášených na webu, ať už jsem to zkoušel s jakýmkoliv hlasování. Z toho důvodu je počet omluvených počítáno z tabulky \textbf{omluvy}, kde na základě atributů \textbf{den}, \textbf{od} a \textbf{do} zjistíme, zda datum a čas omluvy poslance spadá do data a času daného hlasování.

Podle popisu tabulky na webu data slouží pro nahrazení výsledku typu '@', tj. pokud výsledek hlasování jednotlivého poslance je nepřihlášen. Pokud čas hlasování spadá do časového intervalu omluvy, pak se za výsledek považuje 'M', tj. omluven.

\newpage

\begin{center}
	\begin{longtable}{|l|l|p{7cm}|}
		\caption{Tabulka omluvy} 
		\label{table:omluvy} \\
		
		\hline 
		
		\multicolumn{3}{|l|}{\textbf{Tabulka omluvy}} \\
		
		\hline 
		
		\multicolumn{1}{|l|}{\textbf{Sloupec}} & \multicolumn{1}{l|}{\textbf{Typ}} & \multicolumn{1}{l|}{\textbf{Použití a vazby}} \\ 
		
		\endhead
		
		\hline 
		
		id\textunderscore organ & int & Identifikátor volebního období, viz organy:id\textunderscore organ
		\\
		
		\hline 
		
		id\textunderscore poslanec & int & Identifikátor poslance, viz poslanec:id\textunderscore poslanec
		\\
		
		\hline 
		
		
		den & date & Datum omluvy
		\\
		
		\hline 
		
		
		od & datetime(hour to minute)	 & Čas začátku omluvy, pokud je \lstinline|null|, pak i omluvy:do je \lstinline|null| a jedná se o omluvu na celý jednací den.
		\\
		
		\hline 
		
		
		do & datetime(hour to minute)	 & Čas konce omluvy, pokud je \lstinline|null|, pak i omluvy:od je \lstinline|null| a jedná se o omluvu na celý jednací den.	\\
		
		\hline 
		
	\end{longtable}
\end{center}