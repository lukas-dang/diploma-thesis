\chapter{Analýza zdrojových dat}
\label{ch:analysis_data}

\begin{chapterabstract}
	V této kapitole budou analyzována zdrojová data PSP, která budou použita pro backend.
\end{chapterabstract}

\section{Zdrojové soubory}
\label{sec:source-data}

Zdrojové soubory se nachází v souborech ve formátu zip, které lze stáhnout na webu PSP \cite{psp-data}. Pro tuto práci budou použity následující zdrojové soubory:

\begin{itemize}
	\item \textbf{poslanci.zip} - Obsahuje datové soubory pro poslance a osoby, jejich zařazení do orgánů \linebreak a orgány jako takové.
	\item \textbf{hl-XXXXps.zip} -- Obsahuje datové soubory pro návrhy zákonů a výsledky jejich \linebreak hlasování, a výsledky hlasování poslanců.
\end{itemize}

\section{Formát dat}
Data v datových souborech jsou poskytována ve formátu UNL \cite{psp-data}, tj.:

\begin{itemize}
	\item Každý řádek v souboru odpovídá jednom řádku v databázi.
	\item Oddělovačem je znak roury (|).
	\item Pokud je sloupec prázdný, je jeho hodnota typu \lstinline|null|.
	\item V sloupcích jsou používány tzv. escape sekvence k zápisu speciálních znaků s úvodním znakem \ (backslash) následovaný znakem.
\end{itemize}

\noindent Další informace o datech:

\begin{itemize}
	\item Kódování dat je \lstinline|windows-1250|.
	\item Pokud bude strunktura dat doplňována, budou nové sloupce přidávány na konec.
\end{itemize}

\noindent Kódování \lstinline|windows-1250| obsahuje mimo jiné všechny znaky z české abecedy. Je tedy třeba dbát na správně nastavené kódování při parsování nebo ukládání do databáze, aby datům vráceným mobilní aplikaci nechyběly např. háčky a čárky. Zároveň při parsování UNL souborů je potřeba brát ohled na přidávání nových sloupců nakonec.

\section{Aktualizace}
Web PSP \cite{psp-data} uvádí, že data obsahují úplný stav a že rozdílové aktualizace nejsou poskytovány. To je pro použití neideální, jelikož při aktualizaci databáze je pak potřeba z webu stáhnout znovu všechna zdrojová data a poté buď sestavit znovu celou databázi nebo zjistit rozdílový stav a na základě něho aktualizovat stávající databázi.

\section{Datové typy}
Na webu PSP \cite{psp-data} jsou poskytovány informace o datových typech sloupců. V tabulce \ref{table:data_types} jsou uvedeny ty datové typy, které jsou použity v rámci této práce.

\begin{longtable}{|l|p{9cm}|} \hline
	\multicolumn{2}{|l|}{\textbf{Typy dat sloupců v tabulkách}} \\ \hline
	\textbf{Typ} & \textbf{Popis} \\ \hline
	
	int	& integer \\ \hline
	
	char(X)		& textový řetězec, s blíže neuvedenou délkou
	\\ \hline

	date	& datum, ve formátu DD.MM.YYYY
	\\ \hline	
	
	datetime(year to hour)		& datum a čas, do úrovně hodin, ve formátu YYYY-MM-DD HH
	
	\\ \hline
	
	datetime(hour to minute)		& čas, ve formátu HH:MM
	\\ \hline
	
	\caption{Datové typy sloupců v datových souborech \cite{psp-data}}
	\label{table:data_types}
\end{longtable}

\section{Licence}

Data jsou poskytována bezplatně, využití dat je podmíněno uvedením zdroje dat a případně datem zpracování dat \cite{psp-data}. V mobilní aplikaci bude uveden zdroj.

\section{Datové soubory}
\label{sec:data-file}

Zde budou vypsány použité datové soubory ve zdrojových souborech uvedených v podkapitole \ref{sec:source-data} a jejich stručný popis. Soubory budou popsány detailněji v následující sekci \ref{sec:db-tables}. Ze zdrojového souboru \lstinline|poslanci.zip| budou použity následující datové soubory:

\begin{itemize}
	\item \textbf{organy} -- Reprezentuje orgány ve státní správě. Zahrnuje parlamenty všech volebních období a poslanecké kluby. Data budou použita pro nalezení poslaneckých klub.
	
	\item \textbf{poslanec} -- Reprezentuje poslance, včetně těch z předchozích volebních období. Data budou použita pro získání všech poslanců v určitém volebním období.
	
	\item \textbf{osoby} -- Reprezentuje osobu. Obsahuje osobní údaje jako jméno, příjmení a datum narození. Data budou použita pro zjištění osobních údajů o poslanci.
	
	\item \textbf{zarazeni} -- Obsahuje zařazení osoby do nějakého orgánu. Data budou použita pro získání členů poslaneckého klubu.
\end{itemize}

\noindent Ze zdrojových souborů \lstinline|hl-XXXX.zip|, kde XXXX je volební rok, budou používány následující datové soubory:

\begin{itemize}
	\item \textbf{hl\_hlasovani} -- Reprezentuje výsledky hlasování o návrhů zákonů. Data budou použita pro získání informací o návrhu zákona a výsledcích jeho hlasování.
	
	\item \textbf{hl\_poslanec} -- Reprezentuje výsledek hlasování poslance o návrhu zákona. Data budou použita pro získání výsledků hlasování poslance o návrhu zákona.
	
	\item \textbf{omluvy} -- Reprezentuje časově ohraničené omluvy z jednání a hlasování. Data budou použita pro získání počtu omluvených. 
\end{itemize}

\section{Tabulky}
\label{sec:db-tables}

Na diagramu zdrojových dat \ref{fig:class-diagram} jsou ilustrovány tabulky reprezentující datové soubory popsané \linebreak v předchozí sekci \ref{sec:data-file}. Diagram byl vytvořen pomocí nástroje \cite{app-diagram}. Atribut identifikující danou tabulku je zvýrazněn tučným písmem. Některé tabulky jsou identifikovány více atributy. Ve zbytku této sekce budou detailněji popsány jednotlivé tabulky. V tabulkách budou uvedeny pouze ty atributy, kterou jsou použity v rámci této práce. Kompletní výčet atributů lze najít na webu PSP \cite{psp-data}. Většina poznatků ve zbytku této sekce pochází z analýzy obsahu datových souborů a nejsou na stránkách PSP nijak zdokumentována.

\subsection*{organy}
Tabulka \textbf{organy} \ref{table:organy} bude použita pro nalezení poslaneckých klubů v určitém volebním období. Všechny poslanecké kluby mají hodnotu atributu \textbf{id\_typ\_organu} rovnou 1. Poslanecké kluby \linebreak v daném volebním období lze získat dotazem nad všemi kluby, jejichž hodnoty atributů \linebreak \textbf{od\_organ} a \textbf{do\_organ} spadají mezi hodnotami atributů \textbf{od\_organ} a \textbf{do\_organ} PSP. PSP je totiž také orgánem, kterého lze získat přes atribut \textbf{id\_organ}. První PSP má identifikátor 165. PSP v každém následujícím volebním období má hodnotu identifikátoru o jednu větší než PSP \linebreak v předchozím volebním období. Tedy např. PSP z volebního období 2017 až 2021 má identifikátor 172. Dále pokud hodnota atributu \textbf{do\_organ} je rovna \lstinline|null|, pak jde o aktuální orgán.

\begin{center}
	\begin{longtable}{|l|l|p{9cm}|}
		\caption{Tabulka organy} \label{table:organy} \\
		
		\hline 
		
		\multicolumn{3}{|l|}{\textbf{Tabulka organy}} \\
		
		\hline 
		
		\multicolumn{1}{|l|}{\textbf{Sloupec}} & \multicolumn{1}{l|}{\textbf{Typ}} & \multicolumn{1}{l|}{\textbf{Použití a vazby}} \\ 
		
		\endhead
		
		\hline 
		
		\lstinline|id_organ| & \lstinline|int| & Identifikátor orgánu \\
		
		\hline 
		
		\lstinline|id_typ_organu| & \lstinline|int| & Identifikátor typu orgánu \\
		
		\hline 
		
		\lstinline|zkratka| & \lstinline|char(X)| & Zkratka orgánu \\
		
		\hline 
		
		\lstinline|nazev_organu_cz|	 & \lstinline|char(X)|	 & Název orgánu v češtině
		\\
		
		\hline 
		
		\lstinline|od_organ| & \lstinline|date| & Datum ustavení orgánu
		\\
		
		\hline 
		
		\lstinline|do_organ| & \lstinline|date| & Datum ukončení orgánu
		\\
		
		\hline 
		
	\end{longtable}
\end{center}

\subsection*{osoby}

Tabulka \textbf{osoby} \ref{table:osoby} bude použita pro získání osobních údajů o poslancích. Zde je zajímavé datum narození, které je 1. 1. 1900, pokud je neznámo. Zobrazovat toto datum v mobilní aplikaci při absenci data narození poslance by nebylo ideální. V kapitole \ref{sec:rest} je uvedena ztransformovaná verze tohoto data.

\begin{center}
	\begin{longtable}{|l|l|p{9cm}|}
		\caption{Tabulka osoby
		} \label{table:osoby} \\
		
		\hline 
		
		\multicolumn{3}{|l|}{\textbf{Tabulka osoby
		}} \\
		
		\hline 
		
		\multicolumn{1}{|l|}{\textbf{Sloupec}} & \multicolumn{1}{l|}{\textbf{Typ}} & \multicolumn{1}{l|}{\textbf{Použití a vazby}} \\ 
		
		\endhead
		
		\hline 
		
		\lstinline|id_osoba| & \lstinline|int| & Identifikátor osoby \\
		
		\hline 
		
		\lstinline|jmeno| & \lstinline|char(X)| & Jméno \\
		
		\hline 
		
		\lstinline|prijmeni| & \lstinline|char(X)| & Příjmení \\
		
		\hline 
		
		\lstinline|narozeni| & \lstinline|date| & Datum narození, pokud neznámo, pak 1.1.1900. \\
		
		\hline 
		
		\lstinline|pohlavi| & \lstinline|char(X)| & Pohlaví, M jako muž, ostatní hodnoty žena \\
		
		\hline 
		
		
	\end{longtable}
\end{center}

\subsection*{zarazeni}

Tabulka zařazení bude použita ke zjištění příslušnosti poslance do poslaneckého klubu. Atribut \textbf{id\_of} může reprezentovat buď tabulku \textbf{organ} (tu používáme) nebo tabulku \textbf{funkce} (tu nepoužíváme, a tudíž tu není uvedena) podle toho, zda je hodnota atributu \textbf{cl\_funkce} rovna 0 nebo 1. Tabulka \textbf{funkce} reprezentuje konkrétní funkci v daném orgánu. Pro účely této práce tato informace není potřeba, a proto zde tabulka \textbf{funkce} není uvedena. Je však potřeba zjistit příslušnost poslance do poslaneckého klubu, kterou lze zjistit kombinací tabulek \textbf{poslanec}, \textbf{osoby}, \textbf{zarazeni} a \textbf{organy}. Zajímat nás tedy budou pouze taková zařazení, která mají hodnotu atributu \textbf{cl\_funkce} rovnou 0. Dále pokud hodnota atributu \textbf{do\_o} je rovna \lstinline|null|, pak jde o aktuální zařazení.

\begin{center}
	\begin{longtable}{|l|l|p{8cm}|}
		\caption{Tabulka zarazeni} 
		\label{table:zarazeni} \\
		
		\hline 
		
		\multicolumn{3}{|l|}{\textbf{Tabulka zarazeni}} \\
		
		\hline 
		
		\multicolumn{1}{|l|}{\textbf{Sloupec}} & \multicolumn{1}{l|}{\textbf{Typ}} & \multicolumn{1}{l|}{\textbf{Použití a vazby}} \\ 
		
		\endhead
		
		\hline 
		
		\lstinline|id_osoba| & \lstinline|int| & Identifikátor osoby, viz osoba:id\textunderscore osoba \\
		
		\hline 
		
		\lstinline|id_of| & \lstinline|int| & Identifikátor orgánu či funkce: pokud je zároveň nastaveno zarazeni:cl\textunderscore funkce == 0, pak id\textunderscore o odpovídá organy:id\textunderscore organ, pokud cl\textunderscore funkce == 1, pak odpovídá funkce:id\textunderscore funkce.
		\\
		
		\hline 
		
		\lstinline|cl_funkce| & \lstinline|int| & Status členství nebo funkce: pokud je rovno 0, pak jde o členství, pokud 1, pak jde o funkci.
		\\
		
		\hline 
		
		\lstinline|od_o| & \lstinline|datetime(year to hour)|	 & Zařazení od
		\\
		
		\hline 
		
		\lstinline|do_o| & \lstinline|datetime(year to hour)|	 & Zařazení do
		\\
		
		\hline 
		
	\end{longtable}
\end{center}

\subsection*{poslanec}

Tabulka \textbf{poslanec} \ref{table:poslanec} slouží pro získání seznamu poslanců v daném volebním období. Pokud je někdo poslancem ve více volebních obdobích, objeví se v této tabulce vícekrát, pokaždé s jiným identifikátorem. Dále pomocí atributu \textbf{id\_kraj} lze skrz tabulku \textbf{organy} získat název volebního kraje. A pomocí atributu \textbf{id\_obdobi} lze zjistit volební období, v kterém byl poslancem. Hodnota atributu \textbf{id\_obdobi} totiž odpovídá identifikátoru PSP v určitém volebním období. Např. poslanec s hodnotou atributu \textbf{id\_obdobi} rovnou 165 je poslancem PSP s identifikátorem 165. Dále někteří poslanci mají profilové foto. Po analýze fotek poslanců na webu PSP byl zjištěn následující vzor pro URL fotky poslance: \lstinline|https://www.psp.cz/eknih/cdrom/XXXXps/eknih/XXXXps/poslanci/iYYY.jpg|, kde \lstinline|XXXX| je identifikátor osoby a \lstinline|YYY| je identifikátor PSP.

\begin{center}
	\begin{longtable}{|l|l|p{9cm}|}
		\caption{Tabulka poslanec} 
		\label{table:poslanec} \\
		
		\hline 
		
		\multicolumn{3}{|l|}{\textbf{Tabulka poslanec}} \\
		
		\hline 
		
		\multicolumn{1}{|l|}{\textbf{Sloupec}} & \multicolumn{1}{l|}{\textbf{Typ}} & \multicolumn{1}{l|}{\textbf{Použití a vazby}} \\ 
		
		\endhead
		
		\hline 
		
		\lstinline|id_poslanec| & \lstinline|int| & Identifikátor poslance \\
		
		\hline 
		
		\lstinline|id_osoba| & \lstinline|int| & Identifikátor osoby, viz osoba:id\textunderscore osoba \\
		
		\hline 
		
		\lstinline|id_kraj| & \lstinline|int| & Volební kraj, viz organy:id\textunderscore organu \\
		
		\hline 
		
		\lstinline|id_obdobi| & \lstinline|int| & Volební období, viz organy:id\textunderscore organu \\
		
		\hline 
		
		\lstinline|foto| & \lstinline|int| & Pokud je rovno 1, pak existuje fotografie poslance \\
		
		\hline 
		
	\end{longtable}
\end{center}

\subsection*{hl\textunderscore hlasovani}
\label{subsec:hl-hlasovani}

Tabulka \textbf{hl\_hlasovani} obsahuje informace o návrzích zákonů a výsledcích jejich hlasování. Čislo schůze bude použito pro sestavení URL adresu ke stenoprotokolu jednání o daném návrhu zákona. Po analýze stenoprotokolů na webu PSP bylo zjištěno, že lze sestavit URL adresu ke stenoprotokolu ale pouze na první jeho stránky. Nelze sestavit URL adresu na \linebreak stránku, kde se jedná o daném návrhu zákona. Vzor pro URL adresu stenoprotokolu je \linebreak \lstinline|https://www.psp.cz/eknih/XXXXps/stenprot/YYYschuz|, kde \lstinline|XXXX| je volební rok a \lstinline|YYY| je číslo schůze. Dále podle webu se od účinnosti novely jednacího řádu 90/1995 Sb. nerozlišuje zdržel se a nehlasoval, tj. příslušné počty se sčítají. Z toho plyne, že by mobilní aplikaci měla u hlasováních uskutečněných před účinností této novely mít kolonku pro počet poslanců, kterí nehlasovali. A u hlasováních uskutečněných po účinnosti této novely by tam daná kolonka již být neměla.

\newpage

\begin{center}
	\begin{longtable}{|l|l|p{7cm}|}
		\caption{Tabulka hl\textunderscore hlasovani} 
		\label{table:hl_hlasovani} \\
		
		\hline 
		
		\multicolumn{3}{|l|}{\textbf{Tabulka hl\textunderscore hlasovani}} \\
		
		\hline 
		
		\multicolumn{1}{|l|}{\textbf{Sloupec}} & \multicolumn{1}{l|}{\textbf{Typ}} & \multicolumn{1}{l|}{\textbf{Použití a vazby}} \\ 
		
		\endhead
		
		\hline 
		
		\lstinline|id_hlasovani| & \lstinline|int| & Identifikátor hlasování \\
		
		\hline 
		
		\lstinline|schuze| & \lstinline|int| & Číslo schůze
		\\
		
		\hline 
		
		\lstinline|cislo| & \lstinline|int| & Číslo hlasování
		\\
		
		\hline 
		
		\lstinline|datum| & \lstinline|date| & Datum hlasování
		\\
		
		\hline 
		
		\lstinline|cas| & \lstinline|datetime(hour to minute)|	 & Čas hlasování
		\\
		
		\hline 
		
		\lstinline|pro| & \lstinline|int| & Počet hlasujících pro
		\\
		
		\hline 
		
		\lstinline|proti| & \lstinline|int| & Počet hlasujících proti
		\\
		
		\hline 
		
		\lstinline|zdrzel| & \lstinline|int| & Počet hlasujících zdržel se
		\\
		
		\hline 
		
		\lstinline|nehlasoval| & \lstinline|int| & Počet přihlášených, kteří nestiskli žádné tlačítko
		\\
		
		\hline 
		
		\lstinline|prihlaseno| & \lstinline|int| & Počet přihlášených poslanců
		\\
		
		\hline 
		
		\lstinline|vysledek| & \lstinline|char(X)|	 & Výsledek: A -- přijato, R -- zamítnuto, jinak zmatečné hlasování
		\\
		
		\hline 
		
		\lstinline|nazev_dlouhy| & char(X)	 & Dlouhý název bodu hlasování
		\\
		
		\hline 
		
	\end{longtable}
\end{center}

\subsection*{hl\textunderscore poslanec}

Tabulka \textbf{hl\_poslanec} obsahuje výsledky hlasování poslanců o návrzích zákonů, tj. informace \linebreak o tom, jak hlasoval který poslanec o kterém návrhu zákona. Výsledky 'B' a 'N' jsou interpretovány stejně, oba znamenají hlasování proti. Výsledek 'F' znamená, že poslanec nehlasoval. Používal se pouze před rokem 1995, jak bylo popsáno v předchozí sekci \ref{subsec:hl-hlasovani}. Po roce 1995 má vždy hodnotu 0. Výsledek 'W' se v datech vyskytuje velmi málo a v případě, že nastane, se jeho počet přičte k počtu nepřihlášených. U výsledku 'K' se mi nepodařilo zjistit, co přesně znamená \linebreak zdržel se/nehlasoval. V datech se nevyskytuje moc často. Na webu PSP je tento výsledek hlasování u jednoho návrhu zákona v roce 2013 interpretován jako nehlasoval. Z toho lze usoudit pouze to, že 'K' může znamenat, že poslanec nehlasoval. Není však z toho jasné, kdy může znamenat to, že se poslanec zdržel. Na výsledek hlasování to nemá vliv. Na výsledek hlasování má pouze vliv počet pro a proti, a tudíž bylo rozhodnuto, že výsledek 'K' bude ignorován. 

\newpage

\begin{center}
	\begin{longtable}{|l|l|p{9cm}|}
		\caption{Tabulka hl\textunderscore poslanec} 
		\label{table:hl_poslanec} \\
		
		\hline 
		
		\multicolumn{3}{|l|}{\textbf{Tabulka hl\textunderscore poslanec}} \\
		
		\hline 
		
		\multicolumn{1}{|l|}{\textbf{Sloupec}} & \multicolumn{1}{l|}{\textbf{Typ}} & \multicolumn{1}{l|}{\textbf{Použití a vazby}} \\ 
		
		\endhead
		
		\hline 
		
		\lstinline|id_poslanec| & \lstinline|int| & Identifikátor poslance, viz poslanec:id\textunderscore poslanec
		\\
		
		\hline 
		
		\lstinline|id_hlasovani| & \lstinline|int| & Identifikátor hlasování, viz hl\textunderscore hlasovani:id\textunderscore hlasovani
		\\
		
		\hline 
		
		\lstinline|vysledek| & \lstinline|char(X)| & Hlasování jednotlivého poslance. 'A' - ano, 'B' nebo 'N' -- ne, 'C' -- zdržel se, 'F' -- nehlasoval, '@' -- nepřihlášen, 'M' -- omluven, 'W' -- hlasování před složením slibu poslance, 'K' -- zdržel se/nehlasoval.
		\\
		
		\hline 
		
	\end{longtable}
\end{center}

\subsection*{omluvy}

Tabulka \textbf{omluvy} \ref{table:omluvy} zaznamenává časově ohraničené omluvy poslanců z jednání PSP. Slouží pouze po spočtení počtu omluvených poslanců během hlasování. Používá se pro spočítání počtu omluvených poslanců, který v tabulce \ref{table:hl_hlasovani} nejsou. Podle webu PSP slouží tato tabulka pro nahrazení výsledku hlasování typu '@', tj. pokud výsledek hlasování jednotlivého poslance je nepřihlášen. Odůvodnění je, že počet omluvených je přidáván do dat se zpožděním. Dále pokud čas hlasování spadá do časového intervalu omluvy, pak se za výsledek považuje 'M', tj. omluven.

\begin{center}
	\begin{longtable}{|l|l|p{7cm}|}
		\caption{Tabulka omluvy} 
		\label{table:omluvy} \\
		
		\hline 
		
		\multicolumn{3}{|l|}{\textbf{Tabulka omluvy}} \\
		
		\hline 
		
		\multicolumn{1}{|l|}{\textbf{Sloupec}} & \multicolumn{1}{l|}{\textbf{Typ}} & \multicolumn{1}{l|}{\textbf{Použití a vazby}} \\ 
		
		\endhead
		
		\hline 
		
		\lstinline|id_organ| & \lstinline|int| & Identifikátor volebního období, viz organy:id\textunderscore organ
		\\
		
		\hline 
		
		\lstinline|id_poslanec| & \lstinline|int| & Identifikátor poslance, viz poslanec:id\textunderscore poslanec
		\\
		
		\hline 
		
		
		\lstinline|den| & \lstinline|date| & Datum omluvy
		\\
		
		\hline 
		
		
		\lstinline|od| & \lstinline|datetime(hour to minute)|	 & Čas začátku omluvy, pokud je \lstinline|null|, pak \linebreak i omluvy:do je \lstinline|null| a jedná se o omluvu na celý jednací den.
		\\
		
		\hline 
		
		
		\lstinline|do| & \lstinline|datetime(hour to minute)|	 & Čas konce omluvy, pokud je \lstinline|null|, pak \linebreak i omluvy:od je \lstinline|null| a jedná se o omluvu na celý jednací den.	\\
		
		\hline 
		
	\end{longtable}
\end{center}