\chapter{Poslanecká sněmovna}

\begin{chapterabstract}
Táto kapitola slouží jako úvod do tematiky hlasování v poslanecké sněmovně. V první podkapitole bude popsána poslanecká sněmovna jako taková a jakou má roli v politickém sytému ČR. Dále bude stručně popsán hlasovací proces v poslanecké sněmovně. Následně bude popsán webový portál PSP. Na konci bude motivace pro tuto práci.
\end{chapterabstract}

\setcounter{page}{1}

\section{Poslanecká sněmovna}
Základní prvky politického systému ČR představuje prezident, vláda, Parlament a ústavní soud. Ústava ČR dělí moc na zákonodárnou – Parlament, který je složen z Poslanecké sněmovny a Senátu, výkonnou – prezident, vláda a státní zastupitelství a soudní – Ústavní soud a obecné soudy. \cite{Husek2019-p40}

Parlament České republiky se skládá ze dvou komor – Poslanecké sněmovny (dolní komora) a Senátu (horní komora). Poslanecká sněmovna se skládá z 200 poslanců a je volena na čtyři roky na základě poměrného volebního systému \cite{Husek2019-p40}.

\section{Hlasování v poslanecké sněmovně}
Komora PS je usnášeníschopná, pokud je přítomna alespoň jedna třetina jejích členů. K přijetí usnesení (tzn. ke schválení zákona) je nutný souhlas nadpoloviční většiny přítomných poslanců, pokud ústava nestanoví jinak \cite{Husek2019-p40}.

Proces návrhu a schvalování zákona je komplexní a řídí se podle určitých pravidel. Avšak pro účely této práce stačí pochopit schvalovací proces v PS. Kompletní proces přijímání zákonů lze najít na
\href[]{https://www.psp.cz/sqw/hp.sqw?k=173.}{\color{blue}{webu}} PSP.	

\section{Webový portál psp.cz}
Hlavním zdrojem data z PSP je oficiální webový portál \href[]{https://psp.cz/}{\color{blue}{psp.cz}}. Tento portál poskytuje je webová aplikace, ale poskytuje i strojově zpracovatelná data, která budou využita pro tuto práci.

\section{Motivace pro tuto práci}
Web PSP obsahuje veškeré informace ohledně hlasováních, nicméně není responzivní a přizpůsobený pro mobilní zařízení, a tudíž pro uživatele mobilních zařízení je web nepoužitelný. Zároveň srovnatelná aplikace na českém trhu práce ještě neexistuje, a tudíž by se uživatelovi v ČR taková aplikaci mohla hodit. V neposlední řadě tato práce podporuje to, abychom měli informované voliče, zajímající se o to, jak jimi volení zástupci hlasují.