\chapter{Motivace a požadavky}

\begin{chapterabstract}
Táto kapitola slouží jako úvod do tematiky hlasování v poslanecké sněmovně. V první podkapitole bude popsána poslanecká sněmovna jako taková a jakou má roli v politickém sytému ČR. Dále bude stručně popsán hlasovací proces v poslanecké sněmovně. Následně bude popsán webový portál PSP. Po té bude uvedena motivace pro tuto práci. Na konci bude uveden seznam požadavků kladené na mobilní aplikaci a backend.
\end{chapterabstract}

\section{Poslanecká sněmovna}
Základní prvky politického systému ČR představuje prezident, vláda, Parlament a ústavní soud. Ústava ČR dělí moc na zákonodárnou – Parlament, který je složen z Poslanecké sněmovny \linebreak a Senátu, výkonnou – prezident, vláda a státní zastupitelství a soudní – Ústavní soud a obecné soudy.

Parlament České republiky se skládá ze dvou komor – Poslanecké sněmovny (dolní komora) \linebreak a Senátu (horní komora). Poslanecká sněmovna se skládá z 200 poslanců a je volena na čtyři roky na základě poměrného volebního systému.

\section{Hlasování v poslanecké sněmovně}
Komora PS je usnášeníschopná, pokud je přítomna alespoň jedna třetina jejích členů. K přijetí usnesení (tzn. ke schválení zákona) je nutný souhlas nadpoloviční většiny přítomných poslanců, pokud ústava nestanoví jinak.

Proces návrhu a schvalování zákona je komplexní a řídí se podle určitých pravidel. Avšak pro účely této práce stačí pochopit schvalovací proces v PSP. Kompletní proces přijímání zákonů lze najít na webu PSP \cite{psp-vote-process}.

\section{Webový portál psp.cz}
Hlavním zdrojem data z PSP je oficiální webový portál PSP \cite{psp}. Tento portál poskytuje je webová aplikace, ale poskytuje i strojově zpracovatelná data, která budou využita pro tuto práci.

\section{Motivace pro tuto práci}
Web PSP obsahuje veškeré informace ohledně hlasováních, nicméně není responzivní a přizpůsobený pro mobilní zařízení, a tudíž pro uživatele mobilních zařízení je web nepoužitelný. Zároveň srovnatelná aplikace na českém trhu práce ještě neexistuje, a tudíž by se uživatelovi v ČR taková aplikaci mohla hodit. V neposlední řadě tato práce podporuje to, abychom měli informované voliče, zajímající se o to, jak jimi volení zástupci hlasují.

\section{Požadavky}

V této sekci budou popsány funkční a nefunkční požadavky na mobilní aplikaci a backendu. Funkční požadavky specifikují funkcionality, které by měl daný software poskytovat. Nefunkční požadavky určují omezení kladená na daný software. Funkční a nefunkční požadavky budou uvedeny ve formě tabulek. Každá tabulka bude mít dva sloupce. Jedne pro identifikátor daného požadavku pro jednodušší identifikaci v rámci práce, a jeden pro popis daného požadavku.

\subsection{Funkční požadavky}
Tato podkapitola obsahuje seznam funkčních požadavků pro mobilní aplikaci (\ref{table:func_req_app}) a backend (\ref{table:func_req_be}).

\def\arraystretch{1.5}
\begin{longtable}{|l|p{9cm}|} \hline
	\multicolumn{2}{|l|}{\textbf{Funkční požadavky pro mobilní aplikaci}} \\ \hline
	
	\textbf{ID požadavku} & \textbf{Popis požadavku} \\ \hline
	
	FP\textunderscore01	& Aplikace bude umět zobrazit seznam návrhů zákona. Návrh bude obsahovat popis a výsledek jeho hlasování, a datum a čas hlasování. \\ \hline
	
	FP\textunderscore02	& Aplikace bude umět zobrazit detail návrhu zákona. Detail bude zahrnovat název, datum a čas, odkaz na stenoprotokol a oficiální zdroj, a celkovou statistiku hlasování. Celkovou statistikou hlasování rozumíme počet hlasování pro a proti, a počet nepřihlášených, omluvených a zdržených. Dále bude obsahovat hlasování jednotlivých poslaneckých klubů a jejich členů pro daný návrh. \\ \hline
	
	FP\textunderscore03	& Aplikace bude umět zobrazit seznam členů poslanecké sněmovny. V seznamu budou tyto informace o členech: jméno a příjmení, volební kraj, název klubu a profilová fotku. \\ \hline
	
	FP\textunderscore04	& Aplikace bude umět zobrazit detail poslance. Detail poslance bude obsahovat jméno a příjemní, datum narození, profilovou fotku, odkaz na oficiální zdroj, datum nabytí statusu poslance, poslanecký klub a volební kraj. Dále bude obsahovat informace o tom, jak daný poslanec hlasovalv jednotlivých návrzích zákona. \\ \hline
	
	FP\textunderscore05	& Aplikace bude poskytovat možnost nastavení staršího volebního období. Uživatel vždy uvidí návrhy zákonů a seznam poslanců z aktuálně nastavenéího volebního období.\\ \hline
	
	FP\textunderscore06	& Aplikace bude poskytovat možnost vyhledávání hlasování podle jeho popisu.\\ \hline
	
	FP\textunderscore07	& Aplikace bude poskytovat možnost vyhledávání poslance podle jeho jména.\\ \hline
	
	FP\textunderscore07	& Aplikace bude pro seznamy návrhů a poslanců umožňovat tyto seznam nějakým způsobem aktualizovat skrz uživatelské rozhraní.\\ \hline
	
	\caption{Funkční požadavky pro mobilní aplikaci.}
	\label{table:func_req_app}
\end{longtable}

\def\arraystretch{1.5}
\begin{longtable}{|l|p{9cm}|} \hline
	\multicolumn{2}{|l|}{\textbf{Funkční požadavky pro backend}} \\ \hline
	\textbf{ID požadavku} & \textbf{Popis požadavku} \\ \hline
	
	FP\textunderscore03	& Backend bude poskytovat endpoint pro seznam všech volebních období.  \\ \hline
	
	FP\textunderscore01	& Backend bude poskytovat endpoint pro seznam návrhů a výsledků hlasování.  \\ \hline
	
	FP\textunderscore02	& Backend bude poskytovat endpoint pro detail
	návrhu a výsledku hlasování.  \\ \hline
	
	FP\textunderscore03	& Backend bude poskytovat endpoint pro výsledky hlasování poslaneckých klubů a jejich členů pro daný návrh.  \\ \hline
	
	FP\textunderscore04	& Backend bude poskytovat endpoint pro seznam poslanců.  \\ \hline
	
	FP\textunderscore05	& Backend bude poskytovat endpoint pro detail poslance.  \\ \hline
	
	FP\textunderscore06	& Backend bude poskytovat endpoint pro informace o tom, jak hlasoval konkrétní poslanec v jednotlivých návrzích zákona.  \\ \hline
	
	\caption{Funkční požadavky pro backend.}
	\label{table:func_req_be}
\end{longtable}

\subsection{Nefunkční požadavky}

Tato podkapitola obsahuje seznam nefunkčních požadavky pro mobilní aplikaci (\ref{table:nonfunc_req_app}) a backend (\ref{table:nonfunc_req_be}).

\def\arraystretch{1.5}
\begin{longtable}{|l|p{9cm}|} \hline
	\multicolumn{2}{|l|}{\textbf{Nefunkční požadavky pro mobilní aplikaci}} \\ \hline
	\textbf{ID požadavku} & \textbf{Popis požadavku} \\ \hline
	
	NP\textunderscore 01	& Aplikace nebude provádět výpočetně náročná zpracování dat pro šetření aplikace. Výpočetně náročna pracování budou delegována na backend. \\ \hline
	
	NP\textunderscore 02 & Aplikace bude mít jednoduché a intuitivní uživatelské rozhraní. \\ \hline
	
	NP\textunderscore 03 & Aplikace bude fungovat na zařízeních s OS Android 5.1 a výš pro cílení většího množství uživatelů. \\ \hline
	
	NP\textunderscore 04 & Aplikace nebude sbírat uživatelská data. \\ \hline
	
	\caption{Nefunkční požadavky pro mobilní aplikaci.}
	\label{table:nonfunc_req_app}
\end{longtable}

\def\arraystretch{1.5}
\begin{longtable}{|l|p{9cm}|} \hline
	\multicolumn{2}{|l|}{\textbf{Nefunkční požadavky pro backend}} \\ \hline
	\textbf{ID požadavku} & \textbf{Popis požadavku} \\ \hline
	
	NP\textunderscore 01 & Backend bude data stahovat z oficiální portálu PSP. \\ \hline
	
	NP\textunderscore 02 & Backend bude data stažená z portálu PSP transformovat do databázového modelu (\ref{sec:database_model}). Cílem je, aby data byla předzpracovaná a připravená pro rychlý přistup z mobilní aplikace. \\ \hline
	
	NP\textunderscore 03 & Backend bude ztransformovaná data perzistentně ukládat do databáze. \\ \hline
	
	NP\textunderscore 04 & Backend bude data z databáze vystavovat prostřednictvím REST API \cite{rest-api}. \\ \hline
	
	NP\textunderscore 03 & Backend bude každý den stahovat nová data z portálu PSP a aktualizovat databázi. \\ \hline
	
	NP\textunderscore 05 & Backend bude data vystavovat ve formatu JSON. \\ \hline	
	
	
	NP\textunderscore06	& Část dat, jejichž zpracování by trvalo příliš dlouho (např. půl dne) lze zpracovat až za běhu. \\ \hline
	
	\caption{Nefunkční požadavky pro backend.}
	\label{table:nonfunc_req_be}
\end{longtable}
